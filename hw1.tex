\documentclass[11pt]{article}
\usepackage{fullpage,amsthm,amsfonts,amssymb,epsfig,amsmath}

\begin{document}
	\begin{flushleft}
		David Sun\newline
		CMPS 102\newline
		Homework 1\newline

		\item \textbf {\underline{Problem 1:}} 
		In a binary tree all nodes are either internal or they are leaves.  In our definition, internal
		nodes always have two children and leaves have zero children.  Prove that for such trees, the
		number of leaves is always one more than the number of internal nodes.\newline
		
		\textbf{Proof:}\newline
		If \emph{T} is a tree containing \emph{n} internal nodes, then \emph{T} contains \emph{n} + 1 leaves\newline\newline
		\textbf{I. Base Case}\newline
		If \emph{T} has just one internal node, then \emph{T} has two leaves, thus the number of leaves in \emph{T} is one greater than the number of internal nodes in \emph{T}.\newline\newline
		\textbf{II. Induction Step}\newline
		Let \emph{n} $\geq 1$. Assume that all trees containing \emph{n} internal nodes contain \emph{n} + 1 leaves. Generate another internal node on \emph{T} by taking an arbitrary leaf node in \emph{T} and attaching two leaf nodes to it. By generating another internal node, \emph{T} now contains \emph{n} + 1 internal nodes. By our induction hypothesis, \emph{T} now contains (\emph{n} + 1) + 1 leaf nodes. In the process of generating a new internal node, we eliminated a leaf node and attached two additional leaf nodes to \emph{T}, thus \newline ((\emph{n} + 1) - 1) + 2 = (\emph{n} + 1) + 1, as required. 
		\vspace{.2cm}
		
		\item \textbf {\underline{Problem 2:}}
		For $n\ge0$ consider $2^n \times 2^n$ matrices of 1s and 0s in
		which all elements are 1, except one 
		which is 0 (The 0 is at an arbitrary position). 
		Operation: At each step, we can replace three 1s forming an
		"L" with three 0s (The L's can have an arbitrary orientation).\newline
		
		\textbf{Proof:}\newline
		If M is a $2^n \times 2^n$ matrix consisting of all 1s and one 0, there exists a sequence of "L" operations such that replacing three 1s in an "L" pattern in M will give us the 0 matrix. \newline
		
		\textbf{I. Base Case}\newline
		\emph{n} = 0. A $2^0 \times 2^0$ matrix is a 1 $\times$ 1 matrix. The only element can be a 0, and thus applying the "L" operation 0 times is sufficient to obtain the 0 matrix.\newline 
		
		\textbf{II. Induction Step}
		Let $n \ge 0$. Let M be a $2^n \times 2^n$ matrix where M contains exactly one 0 and the rest 1s. Assume there exists a sequence of "L" operations to obtain a 0 matrix for M. For a $2^{n+1} \times 2^{n+1}$ matrix M', we can divide M' into four quadrants, with each of the quadrants containing 
		\vspace{0.2cm}
		
	\end{flushleft}
\end{document}