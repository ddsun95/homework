\documentclass[11pt]{article}
\usepackage{fullpage,amsthm,amsfonts,amssymb,epsfig,amsmath,soul}

\begin{document}
\begin{flushleft}
	David Sun 	\newline
	CMPS 102	\newline
	Homework 3	\newline
	
	\item \textbf {\underline{Problem 1:}} \newline
	\textbf{1a)} Show by induction that $(H_k)^2=2^k I_k$, where
	$I_k$ is the identity matrix of dimension $2^k$. 
	\newline
	\newline
	\textbf {Base Case:} $k = 0$. Then $H_0^2$ = [1][1] = [1] = $2^0 \cdot I_0$.
	\newline
	\newline
	\textbf {Induction Step:} Let $n \geq 0$ and $0 \leq n < k$. Suppose that $(H_n)^2 = 2^n I_n$, where $I_n$ is the identity matrix of dimension $2^n$. 
	\newline
	Then $(H_k)^2$ = 	
	$\begin{bmatrix}
		H_{k - 1} &  H_{k - 1} \\
		H_{k - 1} & -H_{k - 1}
	\end{bmatrix}$
	$\begin{bmatrix}
		H_{k - 1} &  H_{k - 1} \\
		H_{k - 1} & -H_{k - 1}
	\end{bmatrix}$
	=
	$\begin{bmatrix}
		2 \cdot H_{k - 1}^2 &  H_{k - 1}^2 - H_{k - 1}^2  \\
		H_{k - 1}^2 - H_{k - 1}^2 & 2 \cdot H_{k - 1}^2
	\end{bmatrix}$	
	\newline
	\newline
	Let $I_{k - 1}$ denote the identity matrix of dimension $2^{k - 1}$ and $0_{k - 1}$ denote the 0 matrix of dimension $2^{k - 1}$.
	By the induction hypothesis, $2(H_{k - 1})^2$ can be simplified as $2 \cdot 2^{k - 1} I_{k - 1}$. $H_{k - 1}^2 - H_{k - 1}^2$ can be simplified down to $0_{k - 1}$ since the difference of any matrix with itself is the 0 matrix. Thus $(H_k)^2$ can be simplified as
	\newline
	\newline
	$\begin{bmatrix}
		2 \cdot 2^{k - 1} I_{k - 1} & 0_{k - 1} \\
		0_{k - 1} & 2 \cdot 2^{k - 1} I_{k - 1}
	\end{bmatrix}$	
	=
	$\begin{bmatrix}
		2^k I_{k - 1} & 0_{k - 1} \\
		0_{k - 1} & 2^k I_{k - 1}
	\end{bmatrix}$		
	= $2^k$
	$\begin{bmatrix}
		I_{k - 1} & 0_{k - 1} \\
		0_{k - 1} & I_{k - 1}
	\end{bmatrix}$	
	\newline
	Notice that two identity matrices of dimensions $k - 1 \times k- 1$ lie within the main diagonal, and long the antidiagonal are two 0 matrices of dimensions $k - 1 \times k - 1$. This means that the matrix itself the matrix itself is the identity matrix with dimensions $2^k \times 2^k$. 
	\newline
	Thus, 
	$2^k$
	$\begin{bmatrix}
		I_{k - 1} & 0_{k - 1} \\
		0_{k - 1} & I_{k - 1}
	\end{bmatrix}$	
	=
	$2^k I_k$ as required.
	\newline
	\newline
	\textbf {1b)} Note that Hadamard matrices are symmetric, i.e.  $H_k=H_k^\top$. 
	Thus by the above, $H_k H_k^\top = 2^k I_k$ as well. 
	Use this fact for deriving a formula for the dot product
	between the $i$-th and $j$-th row of $H_k$, for $1\le i,j\le 2^k$.
	\newline
	
	The dot product between the $i$-th and $j$-th is 0 whenever $i \neq j$ and the sum of the squares of all the matrix entries whenever $i = j$. In other words, the dot product between any two rows $i$ and $j$ is defined by the following summation:
	\newline
	
	Dot product between the $i$th and $j$th row of $H$ = 
	$\begin{cases} 
		i \neq j & 0 \\
		i    = j & $$\sum_{j=1}^{2^k} H_{ij}^2$$
	\end{cases}$

	\newpage
	\item \textbf {\underline{Problem 2:}} Consider the Coin Changing problem with the European coin set:
	\begin{displaymath}
	\{ 1, 2, 5, 10, 20, 50, 100, 200 \}.
	\end{displaymath}
	Prove that the Cashier's Algorithm is optimal given the above set of coins.
	Use the same proof method that was used for the American coin set in class.
	\newpage
	\item \textbf {\underline{Problem 3:}} Given a sorted array of distinct integers $A[1, . . . , n]$, you want to
	find out whether there is an index $i$ for which $A[i] = i$. Give a
	divide-and-conquer algorithm that runs in time $O(\log n).$
	\newline
	\textbf{Algorithm: } Our algorithm will be called findIndex(A, low, high).
\end{flushleft}
\end{document}