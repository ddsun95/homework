\documentclass[11pt]{article}
\usepackage{fullpage,amsthm,amsfonts,amssymb,epsfig,amsmath,soul}

\begin{document}
	\begin{flushleft}
		David Sun	\newline
		CMPS 102	\newline
		Homework 2	\newline
		\item \textbf {\underline{Problem 1:}} Design 3 algorithms based on binary min heaps that find the $k$th smallest \# out of a set of $n$ \#'s in time:
		\begin{enumerate}
			\item[a)] $O(n \log k)$
			\item[b)] $O(n + k \log n)$
			\item[c)] $O(n + k \log k)$
		\end{enumerate}
		\textbf{Note:} For all problems, an array will be used to represent the heap. \newline
		\textbf{a)} For part a, we initially allocate an array of size $k$ for our heap with all the elements initialized to $-\infty$ \emph{(according to CLRS, a heap containing all duplicate keys can still be classified as a valid min-heap).}
		Then, we will make a pass through our array $A$ from $A[1 \dots length[A]]$. For each $A_i$ where $i \in$ $\lbrace 1 \dots length[A] \rbrace$, check if $-A_i$ is greater than the minimum of the heap. If such is the case, call remove on the heap, and insert $-A_i$ into the heap. We apply this operation for $length[A]$ iterations. Once we've scanned through all the elements in $A$, the $kth$ smallest element should be $-1 * smallest(heap)$.
		\newline
		\textbf{Correctness for a:} The constraint in part a is that only the smallest element can be removed. In this case, we construct a heap using the negatives of the array elements. Thus if the $negative$ of an array element is $greater$ $than$ the minimum heap value, then replace the minimum heap value with the negative of the array element. Since $\forall a, b \in \mathbb{R}$, if $a > b$ then $-b > -a$, replacing the minimum value in the heap containing negatives is equivalent of replacing the maximums in a max-heap. By the end of one pass through the array, the min-heap would contain the negatives of the k-smallest values with the k-th smallest of the array as the smallest value of the heap. To obtain the minimum, multiply the smallest value of the heap by -1 and that value would be the original $k$th smallest element of the array.
		\newline
		\textbf{Runtime analysis for a:} One pass through the array is an $O$(n) operation. Each array element comparison with the smallest heap element is an $O(1)$ operation. In the worst case, we would need to replace the minimum heap at element at every compare, which would cost us $2 \log k$ for a remove and insert operation. Thus, in the worst case, the algorithm generally runs in $n\cdot 2\log k$ time, which suggests the algorithm is $O(n \log k)$
		\newline
		\newline
		\textbf{b)} Call buidheap on the array $A$. This ensures $A$ is now an array that satisfies the heap property. Since we want the $k$th smallest element, call remove $k - 1$ times. After $k - 1$ removes, the $k$th smallest element would be the smallest element on the heap. 
		\newline
		\textbf{Correctness for b:} If $A$ is a proper min-heap containing the original array elements, then a call to remove ensures the next smallest element is at the top of the heap. For $j$ removes on the heap, the $j + 1^{th}$ smallest element is at the top of the heap. Thus, calling remove $k - 1$ times extracts the $(k - 1) + 1^{th}$ smallest element, which is the $k$th smallest. 
		\newline
		\textbf{Runtime analysis for b:} Buildheap on an array of size $n$ is an $O(n)$ 	operation. After buildheap executes, calling remove on a heap of size $n$ $k - 1$ times costs $(k - 1) \log n$ time. In general, the algorithm runs in
		$n + (k - 1) \log n$ time, which means that the algorithm is $O(n + k \log n)$.
		\newpage
		\textbf{c)}
		
	\end{flushleft}
\end{document}