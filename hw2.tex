\documentclass[11pt]{article}
\usepackage{fullpage,amsthm,amsfonts,amssymb,epsfig,amsmath,soul}

\begin{document}
	\begin{flushleft}
		David Sun	\newline
		CMPS 102	\newline
		Homework 2	\newline
		\item \textbf {\underline{Problem 1:}} Design 3 algorithms based on binary min heaps that find the $k$th smallest \# out of a set of $n$ \#'s in time:
		\begin{enumerate}
			\item[a)] $O(n \log k)$
			\item[b)] $O(n + k \log n)$
			\item[c)] $O(n + k \log k)$
		\end{enumerate}
		\textbf{Note:} For all problems, an array will be used to represent the heap. \newline
		\textbf{a)} For part a, we initially allocate an array of size $k$ for our heap with all the elements initialized to $-\infty$ \emph{(according to CLRS, a heap containing all duplicate keys can still be classified as a valid min-heap).}
		Then, we will make a pass through our array $A$ from $A[1 \dots length[A]]$. For each $A_i$ where $i \in$ $\lbrace 1 \dots length[A] \rbrace$, check if $-A_i$ is greater than the minimum of the heap. If such is the case, call remove on the heap, and insert $-A_i$ into the heap. We apply this operation for $length[A]$ iterations. Once we've scanned through all the elements in $A$, the $kth$ smallest element should be $-1 * smallest(heap)$.
		\newline
		\textbf{Correctness for a:} The constraint in part a is that only the smallest element can be removed. In this case, we construct a heap using the negatives of the array elements. Thus if the $negative$ of an array element is $greater$ $than$ the minimum heap value, then replace the minimum heap value with the negative of the array element. Since $\forall a, b \in \mathbb{R}$, if $a > b$ then $-b > -a$, replacing the minimum value in the heap containing negatives is equivalent of replacing the maximums in a max-heap. By the end of one pass through the array, the min-heap would contain the negatives of the k-smallest values with the k-th smallest of the array as the smallest value of the heap. To obtain the minimum, multiply the smallest value of the heap by -1 and that value would be the original $k$th smallest element of the array.
		\newline
		\textbf{Runtime analysis for a:} One pass through the array is an $O$(n) operation. Each array element comparison with the smallest heap element is an $O(1)$ operation. In the worst case, we would need to replace the minimum heap at element at every compare, which would cost us $2 \log k$ for a remove and insert operation. Thus, in the worst case, the algorithm generally runs in $n\cdot 2\log k$ time, which suggests the algorithm is $O(n \log k)$
		\newline
		\newline
		\textbf{b)} Call buidheap on the array $A$. This ensures $A$ is now an array that satisfies the heap property. Since we want the $k$th smallest element, call remove $k - 1$ times. After $k - 1$ removes, the $k$th smallest element would be the smallest element on the heap. 
		\newline
		\textbf{Correctness for b:} If $A$ is a proper min-heap containing the original array elements, then a call to remove ensures the next smallest element is at the top of the heap. For $j$ removes on the heap, the $j + 1^{th}$ smallest element is at the top of the heap. Thus, calling remove $k - 1$ times extracts the $(k - 1) + 1^{th}$ smallest element, which is the $k$th smallest. 
		\newline
		\textbf{Runtime analysis for b:} Buildheap on an array of size $n$ is an $O(n)$ 	operation. After buildheap executes, calling remove on a heap of size $n$ $k - 1$ times costs $(k - 1) \log n$ time. In general, the algorithm runs in
		$n + (k - 1) \log n$ time, which means that the algorithm is $O(n + k \log n)$.
		\newpage
		\textbf{c)} Call buildheap on the array, ensuring the array is now a valid min-heap, call this heap $A$. Allocate another array of size $k$ for our second heap. Let $H$ be a heap containing a key-value pair, with its keys being the elements of $A$ and the values being the index at which the element is found. Initially, insert the smallest of $A$ along with the index of the smallest into $H$. At each iteration, remove the smallest value-index pair $(A[i], i)$ from $H$ and insert $(A[2i], 2i)$ and $(A[2i + 1], 2i + 1)$ into H \emph{(Note: if $length[A]$ $<$ $2i$, do not insert either pair into $H$. If $length[A]$ $<$ $2i + 1$, insert only $(A[2i], 2i)$ into $H$)}. After $k - 1$ iterations, the $k$th smallest element is the smallest key in $H$. 
		\newline
		\textbf{Correctness for c:} By replacing the smallest in our new heap and inserting the children of the smallest from the original heap, all possible candidates for the next minimum are considered. Once the children are inserted, a new minimum is at the root, whether it was an element already in the heap, or a newly inserted element. Since the heap always reads the smallest in $O(1)$ time, repeating this process $k - 1$ times and reading the smallest after $k - 1$ iterations gives us the $k$th smallest value. 
		
		\textbf{Runtime analysis for c:} If the original array has size $n$, buildheap on the original array is an $O(n)$ operation. After buildheap, we execute $k - 1$ operations of insert and remove on $H$, which has size $k$. In the worst case, one remove and two inserts will be called, which has a runtime of $3 \log k$. In general, the algorithm takes $n + (k - 1) \cdot 3 \log k$ time to execute, which means the algorithm is $O(n + k \log k)$.
		\newline
		\item \textbf {\underline{Problem 2:}} Consider the following sorting algorithm for an array of numbers (Assume the size $n$ of the array is divisible by 3):
		\begin{itemize}
			\item   Sort the initial 2/3 of the array.
			\item   Sort the final 2/3 and then again the initial 2/3.
		\end{itemize}
		\textbf{Correctness:} Use induction on $n$, the size of the array, which we'll call $A$. 
		\newline
		\textbf{Base Case:} $n = 3$. Sorting the initial 2/3 ensures $A[1] \leq A[2]$. Sorting the final 2/3 ensures $A[2] \leq A[3]$, thus ensuring that $A[1] \leq A[3]$ and $A[2] \leq A[3]$. Sorting the initial 2/3 ensures again $A[1] \leq A[2]$, ensuring $A[1] \leq A[2] \leq A[3]$.  
		\newline
		\textbf{Inductive Step:} Assume for all $k$, $3 \leq k < n$ that the algorithm properly sorts an array of size $k$. Then, for an array $A$ of size $n$, we can partition up $A$ into three segments B, C, and D, with B containing 
		$A[1 \dots \frac{n}{3}]$, C containing $A[\frac{n}{3} + 1 \dots \frac{2n}{3}]$, and D containing $A[\frac{2n}{3} + 1 \dots n]$. 
		\newline
		Sorting the initial 2/3 of the array sorts B and C. By the inductive hypothesis, B and C are properly sorted and BC is a properly sorted segment with every element in B less than or equal to every element in C.
		\newline
		Sorting the final 2/3 of the array sorts C and D. By the inductive hypothesis, C and D are properly sorted and CD is a properly sorted segment with every element in C less than or equal to every element in D. This ensures that every element in D is greater than or equal to the elements in both B and C. 
		\newline
		Sorting the initial 2/3 of the array sorts B and C again. In the process of sorting CD, C contained the minimal values of D, which could be less than the elements found in B. Thus sorting the the initial 2/3 again ensures every element in B is less than or equal every element in C, by the inductive hypothesis.
		\newline
		Since B, C, D are properly sorted segments and $\forall b \in B$, $\forall c \in C$, $\forall d \in D$, $b \leq c \leq d$, the algorithm properly sorts the array.
		\newline
		\textbf{Runtime analysis:} The algorithm divides the problem into three sub-problems each of size $\frac{2n}{3}$. At each level, there is no other work done so we'll add an $O(1)$ overhead. The recurrence can be written as $T(n) = 3T(\frac{2n}{3}) + O(1)$ or $T(n) = 3T(\frac{n}{1.5}) + O(1)$. We can now apply the master theorem. $a = 3$, $b = 1.5$, $f(n) = n^0$, $k = \log_{1.5}3$. $f(n) = O(n^{k - \epsilon})$ so $T(n) = \Theta(n^{log_{1.5}3})$.
		\newline
		\item \textbf {\underline{Problem 3:}} KT, problem 1, p 246.
		\newline
		\textbf{Algorithm:} Call our two databases D1 and D2. Our divide and conquer algorithm will be called \textbf{findMedian}(D1, D2, lo1, hi1, lo2, hi2), where $lo1$ and $hi1$ are lower and upper bounds for D1 and $lo2$ and $hi2$ are lower and upper bounds for D2. The algorithm will initially be called in the manner \textbf{findMedian}(D1, D2, 1, N, 1, N), if N is the number of elements within each of the databases. The base case of the algorithm will checks if $lo1 == hi1$ and if $lo2 == hi2$. This means we have two sub-lists containing 1 element each, so return $min(D1[lo1],\space D2[lo2])$ to find the median.
		\newline
		If the above comparisons do not hold, let $mid1$ = $\lfloor \frac{lo1 + hi1}{2} \rfloor$
		and let $mid2$ = $\lfloor \frac{lo2 + hi2}{2} \rfloor$. Compare $D1[mid1]$ and $D2[mid2]$. If $D1[mid1] > D2[mid2]$, the median of the two lists lies within the lower half of D1 and in the upper half of D2, so call 
		\textbf{findMedian}(D1, D2, lo1, mid1, mid2, hi2). 
		\newline
		If instead $D1[mid1] < D2[mid2]$ then the median lies of the two lists lies within the upper half of D1, and in the lower half of D2, then call \textbf{findMedian}(D1, D2, mid1, hi1, lo1, mid2). 
		\newline
		\newline
		\textbf{Proof of correctness:} If we have two lists containing one element each, then the median would lie at the 1st position, which would be the minimum of the two lists. Otherwise, for databases each with more than 1 element, taking the median of two databases means that there are at least half the elements less than the higher median and half the elements greater than the lower median, suggesting that the median must lie between the lower and higher median. Therefore, it is necessary to discard half of each database until we reach one element for both databases. 
		\newline
		\newline
		\textbf{Runtime analysis:} At each level, half the input size $n$ is discarded. If each database is n/2 in size, then $((n/2) / 2) + ((n/2) / 2) = n/2$. At each level, there are also number comparisons, done in $O(1)$ time. Thus, the recurrence can be written as $T(n) = T(n/2) + O(1)$. Using the master theorem, we can show that $T(n) = O(\log n)$.
		\newline
		\item \textbf {\underline{Problem 4:}} Suppose you are choosing between the following 3 algorithms:
		\begin{enumerate}
			\item Algorithm $A$ solves problems by dividing them into 5 subproblems of half the size, recursively solving each subproblem, and then combining the solutions in linear time.
			\item Algorithm $B$ solves problems of size n by recursively solving 2 subproblems of size $n-1$ and the combining the solutions in constant time.
			\item Algorithm $C$ solves problems of size $n$ by dividing them into nine subproblems of size $n/3$, recursively solving each subproblem, and the combining the solution in $O(n^2)$ time.
		\end{enumerate}
		What are the running times of each of these algs. (in big-O notation), and which would you choose?
		\newline
		\newpage
		\textbf{Runtime analysis:} 
		\newline
		$\bullet$ The recurrence for (A) can be written as $T_a(n) = 5 \cdot T_a(n/2) + O(n)$
		\newline
		$k = \log_2 5$, $f(n) = n^1$, so $f(n) = O(n^{\log_2 5 - \epsilon})$ so
		$T_a(n) = \Theta(n^{\log_2 5})$, thus $T_a(n) = O(n^{\log_2 5})$
		\newline
		$\bullet$ The recurrence for (B) can be written as $T_b(n) = 2 \cdot T_b(n - 1) + O(1)$
		\newline
		In this case, the master theorem does not apply here. However, analyzing this recurrence, we can see that $T_b(n) = 2T_b(n -1) + k$ = $2\cdot(2\cdot T_b(n - 2) + k) +k$ = $2 \cdot (2 \cdot (2 \cdot T_b(n - 3) +k) +k) +k$ = \newline 
		$2 \cdot (2 \dots (T_b(n - n) +k)\dots +k) +k$. Notice that by expanding the recurrence, there are $n - 1$ products of $2s$. Since the $k$ values are constants, $2^{n-1}$ is the dominating term. Thus, we can conclude that the algorithm runs in $O(2^{n})$ time. \newline
		$\bullet$ The recurrence for (C) can be written as $T_c(n) = 9T(n/3) + n^2$. In this case $k = 2$, $p = 0$ so $f(n) = \Theta(n^2)$, so $T_c(n) = \Theta(n^2 \log^{p + 1} n)$, thus $T_c(n) = \Theta(n^2 \log n)$ so $T_c(n) = O(n^2 \log n)$. 
		\newline
		\newline
		Since algorithm B is an exponential, it grows faster than A and C, and we can conclude it is not the most optimal algorithm. We compare $O(n^{\log_2 5})$ with $O(n^2 \log n)$. Note that $\log_2 5$ $\approx$ 2.31. So, $O(n^{\log_2 5})$ = $O(n^2n^{0.31})$. We compare $n^{0.31}$ to $\log n$. $$\lim_{n\to\infty} \frac{\log n}{n^{0.31}} = 0$$. Therefore, algorithm C is the most optimal algorithm. 
		
		\item \textbf {\underline{Problem 5:}}

		\item
		(a) Compute the FFT of the polynomial $1+2x-x^3$ by computing
		the 4 dimensional FFT matrix and multiplying it with the
		coefficient vector $[1\; 2\; 0\; -\!1]^\top$.
		
		The FFT matrix uses powers of a root of unity.
		First determine the appropriate root of unity.
		\item
		(b) Now compute the inverse FFT of the vector $[1\; 2\; 0\; -\!1]^\top$.
		Again find the appropriate matrix and multiply this matrix
		by the vector.
		\item
		(c) Check that the two matrices used above are inverses of each
		other.
		\newline\newline
		The polynomial has four roots of unity, which are $e^{\frac{0 \cdot 2\pi i}{4}}$, $e^{\frac{1 \cdot 2\pi i}{4}}$, $e^{\frac{2 \cdot 2\pi i}{4}}$, and $e^{\frac{3 \cdot 2\pi i}{4}}$. Let $\omega = e^{\frac{1 \cdot 2\pi i}{4}} = e^{\frac{\pi i}{2}}$. 
		
		The FFT matrix can be expressed as \newline
		$\begin{bmatrix}
			1 & 1 & 1 & 1 \\
			1 & \omega & \omega^2 & \omega^3 \\
			1 & \omega^2 & \omega^4 & \omega^6 \\
			1 & \omega^3 & \omega^6 & \omega^9 \\
		\end{bmatrix}$
		=
		$\begin{bmatrix}
			1 & 1 & 1 & 1 \\
			1 & i   & i^2 & i^3 \\
			1 & i^2 & i^4 & i^6 \\
			1 & i^3 & i^6 & i^9 \\
		\end{bmatrix}$
		= 
		$\begin{bmatrix}
			1 & 1 & 1 & 1 \\
			1 & i   & -1 & -i \\
			1 & -1 & 1 & -1 \\
			1 & -i & -1 & i \\
		\end{bmatrix}$
		\newline
		(a) We multiply this matrix by the coefficient vector by the coefficient vector to get the FFT. 	
		$\begin{bmatrix}
			1 & 1 & 1 & 1 \\
			1 & i   & -1 & -i \\
			1 & -1 & 1 & -1 \\
			1 & -i & -1 & i \\
		\end{bmatrix}$
		$\begin{bmatrix}
			1\\
			2\\
			0\\
		   -1\\
		\end{bmatrix}$
		=
		$\begin{bmatrix}
			2\\
			1 + 3i\\
			0\\
			1 - 3i\\
		\end{bmatrix}$
		\newline
		(b) The inverse Fourier matrix can be expressed as 
		\newline
		$f_4^{-1} = \frac{1}{4}$
		$\begin{bmatrix}
			1 & 	 1 & 		1 & 		1  \\
			1 & \omega^{-1} & \omega^{-2} & \omega^{-3}   \\
			1 & \omega^{-2} & \omega^{-4} & \omega^{-6} \\
			1 & \omega^{-3} & \omega^{-6} & \omega^{-9} \\
		\end{bmatrix}$
		=
		$\frac{1}{4}$
		$\begin{bmatrix}
			1 & 1 & 1 & 1 \\
			1 & -i  & -1 & i \\
			1 & -1 & 1 & -1 \\
			1 & i & -1 & -i \\
		\end{bmatrix}$
		\newpage
		We multiply this matrix with the coefficient vector.
		\newline
		$\frac{1}{4}$
		$\begin{bmatrix}
			1 & 1 & 1 & 1 \\
			1 & -i  & -1 & i \\
			1 & -1 & 1 & -1 \\
			1 & i & -1 & -i \\
		\end{bmatrix}$
		$\begin{bmatrix}
			1\\
			2\\
			0\\
			-1\\
		\end{bmatrix}$	
		=
		$\begin{bmatrix}
			\frac{1}{2}\\
			\\
			\frac{1-3i}{4}\\
			\\
			0\\
			\\
			\frac{1+3i}{4}\\
		\end{bmatrix}$	
		\newline
		Thus obtaining the inverse FFT of the vector. 
		\newline
		(c) To check if the two 4x4 matrices are inverses of each other, we multiply them together and check if the product is the identity matrix. In this case \newline
		$\begin{bmatrix}
			1 & 1 & 1 & 1 \\
			1 & i   & -1 & -i \\
			1 & -1 & 1 & -1 \\
			1 & -i & -1 & i \\
		\end{bmatrix}$
		$\begin{bmatrix}
			1/4 & 1/4 & 1/4 & 1/4 \\
			1/4 & -i/4  & -1/4 & i/4 \\
			1/4 & -1/4 & 1/4 & -1/4 \\
			1/4 & i/4 & -1/4 & -i/4 \\
		\end{bmatrix}$
		=
		$\begin{bmatrix}
			1 & 0 & 0 & 0 \\
			0 & 1 & 0 & 0 \\
			0 & 0 & 1 & 0 \\
			0 & 0 & 0 & 1 \\
		\end{bmatrix}$
		\newline
		Thus, we can conclude that the matrices are indeed inverses.
		\item \textbf {\underline{Problem 6:}} Extra Credit
		\newline
		(a) Let A = 
		$\begin{bmatrix}
			a & b\\
			c & d
		\end{bmatrix}$
		therefore AA
		=
		$\begin{bmatrix}
			a^2 + bc & ab + bd\\
			ac +  cd & cb + d^2
		\end{bmatrix}$
		\newline
		Thus the five Multiplications can be expressed as \newline
		$M_1 = a^2$      \newline
	    $M_2 = bc$		 \newline
	    $M_3 = a(b + d)$ \newline
	    $M_4 = c(a + d)$ \newline
	    $M_5 = d^2$ 	 \newline
	    So AA can be expressed as 
		$\begin{bmatrix}
			M_1 + M_2 & M_3\\
			M_4 & M_2 + M_5
		\end{bmatrix}$
		\newline
		(b) \textbf{Proof by counter-example:} Suppose $n > 2$ and $n$ is a power of 2. Let A be a matrix containing four $n/2$ matrices. Thus A can be expressed as \newline
		A = 
		$\begin{bmatrix}
			A_{11} & A_{12}\\
			A_{21} & A_{22}
		\end{bmatrix}$
	\end{flushleft}
\end{document}