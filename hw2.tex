\documentclass[11pt]{article}
\usepackage{fullpage,amsthm,amsfonts,amssymb,epsfig,amsmath,soul}

\begin{document}
	\begin{flushleft}
		David Sun	\newline
		CMPS 102	\newline
		Homework 2	\newline
		\item \textbf {\underline{Problem 1:}} Design 3 algorithms based on binary min heaps that find the $k$th smallest \# out of a set of $n$ \#'s in time:
		\begin{enumerate}
			\item[a)] $O(n \log k)$
			\item[b)] $O(n + k \log n)$
			\item[c)] $O(n + k \log k)$
		\end{enumerate}
		\textbf{Note:} For all problems, an array will be used to represent the heap. \newline
		\newline
		a) For part a, we initially allocate an array of size $k$ for our heap with all the elements initialized to $-\infty$ \emph{(according to CLRS, a heap containing all duplicate keys can still be classified as a valid min-heap).}
		Then, we will make a pass through our array $A$ from $A[1 \dots length[A]]$. For each $A_i$ where $i \in$ $\lbrace 1 \dots length[A] \rbrace$, check if $-A_i$ is greater than the minimum of the heap. If such is the case, call remove on the heap, and insert $-A_i$ into the heap. We apply this operation for $length[A]$ iterations. Once we've scanned through all the elements in $A$, the $kth$ smallest element should be $-1 * smallest(heap)$.
		\textbf{Correctness for a:} 
		
	\end{flushleft}
\end{document}