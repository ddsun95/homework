\documentclass[11pt]{article}
\usepackage{fullpage,amsthm,amsfonts,amssymb,epsfig,amsmath,soul}

\begin{document}
\begin{flushleft}
	David Sun \newline
	CMPS 102 \newline
	Homework 4
	\item\textbf{\underline{Problem 1:}} \newline (a) The chart below show's Dijkstra's algorithm on the graph. The letter next to each weight keeps track of the parent node of the shortest path tree for that iteration. Assume vertex $A$ is the source. 
	
	\begin{center}
		\begin{tabular}{ | p{1cm} | p{1.2cm} | p{1.2cm} | p{1.2cm} | p{1.2cm} | p{1.2cm} | p{1.2cm} | p{1.2cm} | p{1.2cm} | }
			\hline
			$ $ & A & B & C & D & E & F & G & H\\ 
			
			\hline
			A & 0 & 1 (A) & $\infty$ & $\infty$ & 2 (A) & 10 (A) & $\infty$ & $\infty$ \\
			
			\hline
			B & 0 & 1 (A) & 2 (B) & $\infty$ & 2 (A) & 9 (B) & 4 (B) & $\infty$ \\
			
			\hline
			C & 0 & 1 (A) & 2 (B) & 7 (C) & 2 (A) & 9 (B) & 3 (C) & $\infty$ \\
			
			\hline
			E & 0 & 1 (A) & 2 (B) & 7 (C) & 2 (A) & 6 (E) & 3 (C) & $\infty$ \\
			
			\hline
			G & 0 & 1 (A) & 2 (B) & 7 (C) & 2 (A) & 6(E) & 3 (C) & 13 (G)\\
			
			\hline
			F & 0 & 1 (A) & 2 (B) & 7 (C) & 2 (A) & 6 (E) & 3 (C) & 13 (G)\\
			
			\hline
			D & 0 & 1 (A) & 2 (B) & 7 (C) & 2 (A) & 6 (E) & 3 (C) & 10 (D)\\
			
			\hline
			H & 0 & 1 (A) & 2 (B) & 7 (C) & 2 (A) & 6 (E) & 3 (C) & 10 (D)\\
			\hline
		\end{tabular}
	\end{center}
	(b) Show the shortest path tree corresponding to running Dijkstra on
	this graph. 
	\newline
	From the table, we can derive the shortest-path tree by looking at the top-most row and bottom-most row. In this case, the edge set for our shortest-path tree is $\lbrace AB, BC, CD, AE, EF, CG, DH\rbrace$. Thus the tree can be visualized as:
	
	\includegraphics[width=3in]{graph1.png}
	\newpage
	\item\textbf{\underline{Problem 2:}} Trace Kruskal's algorithm on the graph.
	\newline
	First we sort the edges by weight and we construct a tree by taking the edges in order so long as it does not introduce a cycle. 
	\newline
	\begin{tabular}{ | p{2cm} | p{2cm} | p{5cm} | }
		\hline
		Edge & Weight & Take? \\
		

		\hline
		A - E & 1 & Yes \\
		
		\hline
		A - B & 2 & Yes \\
		
		\hline
		B - F & 2 & Yes \\
		
		\hline
		C - D & 3 & Yes \\
		
		\hline
		F - G & 3 & Yes \\
		
		\hline
		E - F & 4 & No, introduces cycle ABEF. \\
		
		\hline
		C - G & 4 & Yes \\
		
		\hline
		B - E & 5 & No, introduces cycle ABE \\
		
		\hline
		C - F & 5 & No, introduces cycle CFG \\
		
		\hline
		D - G & 5 & No, introduces cycle CGD \\
		
		\hline
		G - H & 6 & Yes \\
		
		\hline
		B - C & 7 & No, the tree is spanning \\
		
		\hline
		D - H & 7 & No, the tree is spanning \\
		
		\hline
		
	\end{tabular}
	\newline
	\newline
	Our MST can be visualized as \newline
	\includegraphics[width=4in]{graph2.png}
	\newline
	\item\textbf{\underline{Problem 3:}} \begin{enumerate}
		\item
		Consider the edge $(B,E)$ in the graph of Problem 2.
		Either show that there is a minimum cost spanning tree using
		this edge by applying the cut property, or
		prove that this edge is not used in any minimum cost
		spanning tree using the cycle property.
		\item
		Ditto for edge $(C,G)$.
	\end{enumerate}
	\textbf{Part 1: }Consider the cycle $A - B$, $B - E$, $E - A$. Out of all these edges $B - E$ has maximal weight. By the cycle property, edge $B - E$ is not part of any MST of the graph in problem 2. 
	\newline
	\newline
	\textbf{Part 2:} Consider the cut $\lbrace C, D \rbrace$. The corresponding cut-set is the set of edges \newline $\lbrace$$C - B,$ $C - F,$ $C - G,$ $D - G,$ $H - D \rbrace$. Out of these set of edges, edge $C - G$ has the minimal weight of 4, so by the cut property, the MST of problem 2 must contain edge $C - G$. 
	\newline
	\newpage
	\item\textbf{\underline{Problem 4:}} Prove or disprove the following statements about an
	arbitrary undirected graph $G=(V,E)$:
	

	\begin{enumerate}
		%\item The shortest-path tree computed by Dijkstra’s algorithm is necessarily
		%an MST.
		\item If e is part of some MST of G, then it must be a lightest edge
		in some cutset of G.
		\item If graph G has more than $|V | - 1$ edges, and there is a unique
		heaviest edge, then this edge cannot be part of a minimum spanning
		tree. 
		\item The shortest path between two nodes is necessarily part of some MST.
	\end{enumerate}
	\textbf{Part 1: }Assume towards a contradiction that if $e$ is part of some MST of $G$, then it is not necessarily the lightest edge in some cutset of $G$. Let $CS$ represent the cutset that connects the cut with the rest of $G$. It is given that $e \in CS$ and suppose $\exists  e' \in CS$ such that $weight(e) > weight(e')$. Our claim is that the MST of $G$ currently contains contains $e$ instead of $e'$. The cut property states that if an edge $x$ is the minimum cost edge with exactly one endpoint in a particular cut, then the MST of that graph must contain $x$. However, our MST of G contained $e$, rather than the lower weighted edge, $e'$. Thus by assuming that $e$ is not the lightest weight edge in some cutset of $G$, we have violated the cut property, so $e$ is indeed the lightest edge in some cutset of $G$ when $e$ is in the MST. 
	\newline
	\newline
	\textbf{Part 2: } False, consider the graph $G$ below. \newline
	\includegraphics[width=2in]{graph4.png} and its MST 
	\includegraphics[width=2in]{graph5.png} \newline
	The graph contains 4 vertices and 4 edges, so G has more than $|V| - 1$ edges. The unique heaviest edge is $A - B$. The MST of $G$ still contains $A - B$ because the tree cannot span $G$ without it. 
	\newline
	\newline
	\textbf{Part 3: } False, consider the graph in problem 2. The shortest distance between the nodes $F$ and $C$ is has weight 5 and is the edge $F - C$. However, the MST of the graph in problem 2 does not contain the edge $F - C$.  
	\newpage
	\item\textbf{\underline{Problem 5:}} Consider an undirected graph $G=(V,E)$ with nonnegative
	edge weights $w_e\ge 0$. Suppose that you have computed a
	minimum spanning tree of $G$, and that you have also
	computed shortest paths to all nodes from a particular
	$s\in V$.
	
	Now suppose each edge weight is replaced by $w_e'=f(w_e)$,
	where $f$ is a strictly increasing function.
	\begin{enumerate}
		\item
		Does the minimum spanning tree change? Give an example
		where it changes or prove that it cannot change.
		\item
		Do the shortest paths change? Give an example where they
		change or prove they cannot change.
	\end{enumerate}

	\textbf{Part 1: }The minimum spanning tree does not change. Assume towards a contradiction that when the function to every $e \in E$, that a different MST $M'$ can be constructed where $M'$ contains a different edge from the original MST of $G$. By definition of an MST, the sum of the weights in the original MST is minimized. Note that $f(w_e)$ is strictly increasing. Thus for every $e_1, e_2 \in E$ and $e_1 < e_2$, then $f(e_1) < f(e_2)$. This means that if $M'$ contained a different edge than the original MST, 
	
\end{flushleft}
\end{document}