\documentclass[11pt]{article}
\usepackage{fullpage,amsthm,amsfonts,amssymb,epsfig,amsmath,soul}

\begin{document}
\begin{flushleft}
	David Sun \newline
	CMPS 102 \newline
	Homework 4
	\item\textbf{\underline{Problem 1:}} \newline (a) The chart below show's Dijkstra's algorithm on the graph. The letter next to each weight keeps track of the parent node of the shortest path tree for that iteration. Assume vertex $A$ is the source. 
	
	\begin{center}
		\begin{tabular}{ | p{1cm} | p{1.2cm} | p{1.2cm} | p{1.2cm} | p{1.2cm} | p{1.2cm} | p{1.2cm} | p{1.2cm} | p{1.2cm} | }
			\hline
			$ $ & A & B & C & D & E & F & G & H\\ 
			
			\hline
			A & 0 & 1 (A) & $\infty$ & $\infty$ & 2 (A) & 10 (A) & $\infty$ & $\infty$ \\
			
			\hline
			B & 0 & 1 (A) & 2 (B) & $\infty$ & 2 (A) & 9 (B) & 4 (B) & $\infty$ \\
			
			\hline
			C & 0 & 1 (A) & 2 (B) & 7 (C) & 2 (A) & 9 (B) & 3 (C) & $\infty$ \\
			
			\hline
			E & 0 & 1 (A) & 2 (B) & 7 (C) & 2 (A) & 6 (E) & 3 (C) & $\infty$ \\
			
			\hline
			G & 0 & 1 (A) & 2 (B) & 7 (C) & 2 (A) & 6(E) & 3 (C) & 13 (G)\\
			
			\hline
			F & 0 & 1 (A) & 2 (B) & 7 (C) & 2 (A) & 6 (E) & 3 (C) & 13 (G)\\
			
			\hline
			D & 0 & 1 (A) & 2 (B) & 7 (C) & 2 (A) & 6 (E) & 3 (C) & 10 (D)\\
			
			\hline
			H & 0 & 1 (A) & 2 (B) & 7 (C) & 2 (A) & 6 (E) & 3 (C) & 10 (D)\\
			\hline
		\end{tabular}
	\end{center}
	(b) Show the shortest path tree corresponding to running Dijkstra on
	this graph. 
	\newline
	From the table, we can derive the shortest-path tree by looking at the top-most row and bottom-most row. In this case, the edge set for our shortest-path tree is $\lbrace AB, BC, CD, AE, EF, CG, DH\rbrace$. Thus the tree can be visualized as:
	
	\includegraphics[width=3in]{graph1.png}
	\newpage
	\item\textbf{\underline{Problem 2:}} Trace Kruskal's algorithm on the graph.
	\newline
	First we sort the edges by weight and we construct a tree by taking the edges in order so long as it does not introduce a cycle. 
	\newline
	\begin{tabular}{ | p{2cm} | p{2cm} | p{5cm} | }
		\hline
		Edge & Weight & Take? \\
		

		\hline
		A - E & 1 & Yes \\
		
		\hline
		A - B & 2 & Yes \\
		
		\hline
		B - F & 2 & Yes \\
		
		\hline
		C - D & 3 & Yes \\
		
		\hline
		F - G & 3 & Yes \\
		
		\hline
		E - F & 4 & No, introduces cycle ABEF. \\
		
		\hline
		C - G & 4 & Yes \\
		
		\hline
		B - E & 5 & No, introduces cycle ABE \\
		
		\hline
		C - F & 5 & No, introduces cycle CFG \\
		
		\hline
		D - G & 5 & No, introduces cycle CGD \\
		
		\hline
		G - H & 6 & Yes \\
		
		\hline
		B - C & 7 & No, the tree is spanning \\
		
		\hline
		D - H & 7 & No, the tree is spanning \\
		
		\hline
		
	\end{tabular}
	\newline
	\newline
	Our MST can be visualized as \newline
	\includegraphics[width=4in]{graph2.png}
	\newline
	\item\textbf{\underline{Problem 3:}} \begin{enumerate}
		\item
		Consider the edge $(B,E)$ in the graph of Problem 2.
		Either show that there is a minimum cost spanning tree using
		this edge by applying the cut property, or
		prove that this edge is not used in any minimum cost
		spanning tree using the cycle property.
		\item
		Ditto for edge $(C,G)$.
	\end{enumerate}
	{part 1:} Consider the cycle $A - B$, $B - E$, $E - A$. Out of all these edges $B - E$ has maximal weight. By the cycle property, edge $B - E$ is not part of any MST of the graph in problem 2. 
	\newline
	\newline
	{part 2:} Consider the cut $\lbrace C, D \rbrace$. The corresponding cut-set is the set of edges \newline $\lbrace$$C - B,$ $C - F,$ $C - G,$ $D - G,$ $H - D \rbrace$. Out of these set of edges, edge $C - G$ has the minimal weight of 4, so by the cut property, the MST of problem 2 must contain edge $C - G$. 
	\newline
	\newpage
	\item\textbf{\underline{Problem 4:}} Prove or disprove the following statements about an
	arbitrary undirected graph $G=(V,E)$:
	\begin{enumerate}
		%\item The shortest-path tree computed by Dijkstra’s algorithm is necessarily
		%an MST.
		\item If e is part of some MST of G, then it must be a lightest edge
		in some cutset of G.
		\item If graph G has more than $|V | - 1$ edges, and there is a unique
		heaviest edge, then this edge cannot be part of a minimum spanning
		tree. 
		\item The shortest path between two nodes is necessarily part of some MST.
	\end{enumerate}
	
\end{flushleft}
\end{document}